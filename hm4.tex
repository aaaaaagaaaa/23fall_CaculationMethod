% 计算方法第四周
\documentclass[utf8,a4paper,11pt]{article}
\usepackage{ctex}
\usepackage{graphicx}
\usepackage{amssymb}
\usepackage{amsmath}
\usepackage{mathrsfs}
\usepackage{booktabs}

\title{第四次上机作业}
\author{21300180079吕昂格}
\date{\today}

\begin{document}
\maketitle
\section*{第一题}
对于对原来线性方程组的系数矩阵进行扰动得到的扰动矩阵\(A+\delta A\),
方程\((A+\delta A)(x+\delta x) = b\),用python计算得出(保留三位小数):
\[\delta x = [-10.586, 17.374, -4.226, 2.524]\]
误差的二范数:
\[\| \delta x \Vert = 20.932\]
解的相对误差为:
\[\frac{\| \delta x \Vert}{\| x \Vert} = 0.0955\]
此时系数矩阵的相对误差为:
\[\frac{\| \delta A \Vert}{\| A \Vert} = 0.0076\]

从理论角度分析,我们有如下公式(北大教材定理2.2.1):
\[\frac{\| \delta x \Vert}{\| x \Vert} \leqslant \frac{k(A)}{1-k(A)\frac{\| \delta A \Vert}{\| A \Vert}}\frac{\| \delta A \Vert}{\| A \Vert} \]
其中\(k(A) = \|A^{-1}\Vert \|A\Vert\)
且当\(\frac{\| \delta A \Vert}{\| A \Vert}\)很小时,上面的估计式可化为:
\[\frac{\| \delta x \Vert}{\| x \Vert} \leqslant k(A) \frac{\| \delta A \Vert}{\| A \Vert} \]
本题计算得出的\(k(A)=2984.0927\),得出对解的误差估计为: 
\[\frac{\| \delta x \Vert}{\| x \Vert}\leqslant 25.962\]

\section*{第二题}
构建算法2.5.1的程序见文件\\
得到对Hilbert矩阵的条件数估计:
\begin{table}[!ht]
    \begin{tabular*}{\hsize}{@{}@{\extracolsep{\fill}}cc@{}}
    \toprule
    阶数 &无穷范数条件数\\
    \midrule
    5 & 943655\\
    10 & \(3.54 \times 10^{13}\)\\
    15 & \(7.04\times 10^{17}\)\\
    20 & \(1.35 \times 10^{18}\)\\
    \bottomrule
    \end{tabular*}
  \end{table}
对于问题2我们采用numpy.random库生成随机向量\(x\),与Guass消去法得到的解产生偏差\(\delta x\)

对n从5到30估计相对误差\(\displaystyle\frac{\| \delta x \Vert}{\| x \Vert}\):
\begin{table}[!ht]
    \begin{tabular*}{\hsize}{@{}@{\extracolsep{\fill}}cc@{}}
    \toprule
    阶数 &解相对误差\\
    \midrule
    5 & \(1.78\times 10^{-16}\)\\
    10 & \(1.39\times 10^{-15}\)\\
    15 & \(1.28\times 10^{-14}\)\\
    20 & \(1.90\times 10^{-12}\)\\
    25 & \(4.09\times 10^{-11}\)\\
    30 & \(7.01\times 10^{-9}\)\\
    \bottomrule
    \end{tabular*}
  \end{table}
对于问题2我们采用numpy.ra

\end{document}