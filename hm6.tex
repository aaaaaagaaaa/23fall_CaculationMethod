% 计算方法第五周
\documentclass[utf8,a4paper,11pt]{article}
\usepackage{ctex}
\usepackage{graphicx}
\usepackage{amssymb}
\usepackage{mathrsfs}
\usepackage{booktabs}
\usepackage{amsmath}
\usepackage{underscore}

\title{第六次上机作业}
\author{21300180079吕昂格}
\date{\today}

\begin{document}
\maketitle
\section*{复旦教材第一题}
复现书上例题4,见文件\(hm 6\_ 1.py\),得到结果如下:
\begin{table}[!ht]
    \begin{tabular*}{\hsize}{@{}@{\extracolsep{\fill}}cccc@{}}
    \toprule
    x &\(F(x)\) & \(F^{'}(x)\) & \([F^{'}(x)]^{-1}F(x)\)\\
    \midrule
    1.15      &4.28750000\(\times 10^{-2}\)& -1.0325&    -4.15254237\(\times 10^{-2}\) \\
    1.19152542& 6.02064963\(\times 10^{-3}\)& -0.74080149& -8.12721043\(\times 10^{-3}\)\\
    1.19965263& 2.36643116\(\times 10^{-4}\)&-0.68250067& -3.46729499\(\times 10^{-4}\)\\
    1.19999936&4.32713245\(\times 10^{-7}\)& -0.68000458 &-6.36338720\(\times 10^{-7}\)\\
    \bottomrule
    \end{tabular*}
\end{table}

\section*{复旦教材第二题}
复现书上例题6,见文件\(hm  6\_ 2.py\),得到结果如下:

\begin{table}[!ht]
    \begin{tabular*}{\hsize}{@{}@{\extracolsep{\fill}}cccc@{}}
    \toprule
    x &\(F(x)\) & \(F^{'}(x)\) & \([F^{'}(x)]^{-1}F(x)\)\\
    \midrule
    1.15      &4.28750000\(\times 10^{-2}\)& -1.0325&    -4.15254237\(\times 10^{-2}\) \\
    1.19152542& 6.02064963\(\times 10^{-3}\)& -1.0325& -5.83113765\(\times 10^{-3}\)\\
    1.19735656& 1.82267575\(\times 10^{-3}\)&-1.0325&  -1.76530339\(\times 10^{-3}\)\\
    1.19912186&5.99907314\(\times 10^{-4}\)& -1.0325 &-5.81024032\(\times 10^{-4}\)\\
    \bottomrule
    \end{tabular*}
\end{table}
可见简化Newton迭代法比起Newton迭代法精度更差


\section*{清华教材习题}
我们设迭代函数为:
\[
\Phi (x_1,x_2,x_3)=
\begin{pmatrix}
    \frac{1}{6}+\frac{1}{3}\cos(x_2 x_3)\\
    \frac{1}{9} \sqrt{x_1^2 + \sin x_3 +1.06} - 0.1\\
    -\frac{1}{20}e^{-x_1 x_2} - \frac{1}{20}(\frac{10}{3}\pi -1)\\
\end{pmatrix}
\]
则有:
\[
\Phi ^{'}(x)=
\begin{pmatrix}
    0 & -\frac{1}{3}x_3 \sin x_2 x_3 &-\frac{1}{3}x_2 \sin x_2 x_3\\
    \frac{1}{9} \frac{x_1}{\sqrt{x_1^2 + \sin x_3 +1.06}}&0&\frac{1}{18} \frac{\cos x_3}{\sqrt{x_1^2 + \sin x_3 +1.06}}\\
    \frac{x_2}{20} e^{-x_1 x_2}&\frac{x_1}{20} e^{-x_1 x_2}&0\\
\end{pmatrix}    
\]
在区域\(D=\{(x_1,x_2,x_3)\in \mathbb{R}^3 | \;|x_i| \leq 1\}\)上\(\|\Phi^{'}(x)\Vert _1 \leq 1\),
故\(\rho(\Phi^{'}(x)) \leq 1\)迭代法收敛,用\(x_0=(0, 0, 0)\)计算方程的根(见文件\(hm6\_ 3\_1.py\)):

我们设置最大迭代次数\(N=100\),在迭代到\(\|x_{k+1} - x_{k} \Vert \leq 10^{-5}\)时结束迭代,
计算得到最后的近似解为:\[x = (5.00000000\times 10^{-1}  2.48385686\times 10 ^{-8} -5.23598776\times 10^{-1})\]

接下来我们使用Newton法计算方程的近似解:\\
以\((0.5, 0.5, 0.5)\)为初值,计算的解为:\[(5.00000000\times 10^{-1},2.84752053\times 10^{-18},-5.23598776\times 10{-1})\]
以\((0.5, 0.2, 0.8)\)为初值,计算的解为:\[(5.00000000\times 10^{-1},-1.35134910\times 10^{-18},-5.23598776\times 10{-1})\]
以\((0.1, 0.3, 0.3)\)为初值,计算的解为:\[(5.00000000\times 10^{-1},-6.92132294\times 10^{-19},-5.23598776\times 10{-1})\]

\end{document}