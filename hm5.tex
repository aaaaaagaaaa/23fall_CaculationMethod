% 计算方法第五周
\documentclass[utf8,a4paper,11pt]{article}
\usepackage{ctex}
\usepackage{graphicx}
\usepackage{amssymb}
\usepackage{mathrsfs}
\usepackage{booktabs}
\usepackage{amsmath}

\title{第五次上机作业}
\author{21300180079吕昂格}
\date{\today}

\begin{document}
\maketitle
\section*{第一题}
本题使用线性方程组迭代法近似拟合一个二阶边值问题的常微分方程,迭代矩阵\(A\)已由题目给定,
下面确定常值向量\(b\):

针对0处的边值问题,由于\(y(0) = 0\),因此失去一项\(\epsilon * y_0\)无影响,即可设置\(b[0]=a*h^2\)。
对于1处的边值为\(y(1)=1\),因此在向量b的最后一项设置为\(a*h^2 - (\epsilon+h)\),即:
\[\epsilon *y_{99} -(2\epsilon +h)*y_{100} = ah^2 - (\epsilon +h)\]
接下来再使用求解线性方程组的迭代求法求得原微分方程的近似数值解:\\
使用Jacobi迭代法求得的解:(迭代次数1000次)\\
\(y=[0.0112, 0.0223, 0.0333, 0.0443, 0.0552, 0.0661, 0.077, 0.0878, 0.0985, 0.1092,\\ 
0.1199, 0.1305, 0.1411, 0.1516, 0.1622, 0.1726, 0.1831, 0.1935, 0.2039, 0.2143, \\
0.2246, 0.2349, 0.2452, 0.2555, 0.2658, 0.276, 0.2862, 0.2964, 0.3066, 0.3167, \\
0.3269, 0.337, 0.3471, 0.3572, 0.3672, 0.3773, 0.3873, 0.3974, 0.4074, 0.4174, \\
0.4274, 0.4374, 0.4473, 0.4573, 0.4672, 0.4771, 0.487, 0.4969, 0.5068, 0.5167, \\
0.5265, 0.5363, 0.5462, 0.556, 0.5657, 0.5755, 0.5853, 0.595, 0.6047, 0.6144, \\
0.6241, 0.6337, 0.6434, 0.653, 0.6626, 0.6722, 0.6817, 0.6912, 0.7007, 0.7102, \\
0.7197, 0.7291, 0.7385, 0.7479, 0.7573, 0.7666, 0.7759, 0.7852, 0.7944, 0.8037, \\
0.8129, 0.822, 0.8312, 0.8403, 0.8493, 0.8584, 0.8674, 0.8764, 0.8853, 0.8942, \\
0.9031, 0.912, 0.9208, 0.9296, 0.9383, 0.9471, 0.9557, 0.9644, 0.973, 0.9816]\)
\newpage
使用G-S迭代法求得的解:(迭代次数100次)\\
\(y = [0.0212, 0.0422, 0.063, 0.0836, 0.104, 0.1242, 0.1442, 0.164, 0.1835, 0.2029,\\
0.2219, 0.2408, 0.2593, 0.2777, 0.2957, 0.3135, 0.331, 0.3483, 0.3653, 0.382, \\
0.3984, 0.4146, 0.4304, 0.446, 0.4613, 0.4763, 0.491, 0.5055, 0.5196, 0.5335, \\
0.547, 0.5603, 0.5733, 0.586, 0.5984, 0.6105, 0.6224, 0.6339, 0.6452, 0.6563, \\
0.667, 0.6775, 0.6877, 0.6977, 0.7073, 0.7168, 0.726, 0.7349, 0.7436, 0.7521, \\
0.7603, 0.7683, 0.7761, 0.7837, 0.791, 0.7982, 0.8051, 0.8118, 0.8184, 0.8247, \\
0.8309, 0.8369, 0.8427, 0.8484, 0.8539, 0.8592, 0.8644, 0.8695, 0.8744, 0.8791, \\
0.8838, 0.8883, 0.8927, 0.897, 0.9012, 0.9053, 0.9093, 0.9132, 0.9171, 0.9208, \\
0.9245, 0.9281, 0.9316, 0.9351, 0.9386, 0.9419, 0.9453, 0.9486, 0.9519, 0.9551, \\
0.9583, 0.9615, 0.9647, 0.9679, 0.9711, 0.9742, 0.9774, 0.9806, 0.9837, 0.9869]\)
使用SOR迭代法求得的解:(迭代次数20次)\\
\(y = [0.0106, 0.0211, 0.0316, 0.0421, 0.0526, 0.0631, 0.0735, 0.084, 0.0944, 0.1048, \\
0.1152, 0.1256, 0.136, 0.1464, 0.1567, 0.1671, 0.1774, 0.1877, 0.198, 0.2083, \\
0.2186, 0.2289, 0.2392, 0.2494, 0.2597, 0.27, 0.2802, 0.2904, 0.3007, 0.3109, \\
0.3211, 0.3313, 0.3415, 0.3517, 0.3619, 0.3721, 0.3822, 0.3924, 0.4026, 0.4127, \\
0.4229, 0.433, 0.4432, 0.4533, 0.4635, 0.4736, 0.4837, 0.4938, 0.504, 0.5141, \\
0.5242, 0.5343, 0.5444, 0.5545, 0.5646, 0.5747, 0.5848, 0.5949, 0.605, 0.615, \\
0.6251, 0.6352, 0.6453, 0.6554, 0.6654, 0.6755, 0.6856, 0.6956, 0.7057, 0.7158, \\
0.7258, 0.7359, 0.7459, 0.756, 0.766, 0.7761, 0.7861, 0.7962, 0.8062, 0.8163, \\
0.8204, 0.8292, 0.8381, 0.8469, 0.8557, 0.8645, 0.8732, 0.8818, 0.8904, 0.899, \\
0.9075, 0.916, 0.9244, 0.9327, 0.9411, 0.9494, 0.9576, 0.9658, 0.974, 0.9821]\)

我们用计算出的数值近似解与精确解之间的数列平方范数来估计拟合效果,计算得到:
\[\|y^{*} - y_{Jacobi} \Vert = 0.250068\]
\[\|y^{*} - y_{G-S} \Vert =1.452607\]
\[\|y^{*} - y_{SOR} \Vert =0.256305\]
其中\(SOR\)方法仅仅迭代20次,由此可见\(SOR\)迭代法的收敛速度显著优于Jcobi迭代法和G-S迭代法


\section*{第二题}
我们对x轴和y轴分别做10等分,记\(u_{ij}\)为u在\((i,j)\)分点的值,则有:
\[u_{i-1,j}+u_{i+1,j}+ u_{i,j-1}+u_{i,j+1}-4u_{ij}=(i^2+j^2)e^{\frac{ij}{100}}\]
为五点差分问题的一般形式,我们把它转化为一个100维的线性方程组\(Au = b\),其中矩阵\(A\):
\[\begin{bmatrix}
    -4 & 1 & 0 & 0 & \dots & 1 & 0 & \dots & \dots &\dots\\
    1 & -4 & 1 & 0 & \dots & 0 & 1 & \dots & \dots &\dots\\
    0 & 1 & -4 & 1 & \dots & \dots & 0 & 1 & \dots &\dots\\
    \vdots & \vdots & \ddots & \vdots & \vdots& \vdots& \vdots& \vdots\\
    0 & 0 & 0 &\ddots\\
    \vdots & \vdots & \vdots& &\ddots\\
    1 & 0 & 0 & & &  \ddots\\
    0 & 1 & 0& & & & \ddots\\
    0 & 0 & 1& & & & & \ddots\\
    \dots & \dots & \dots & \dots & \dots & \dots & 0 & 1 & -4 \\

\end{bmatrix}\]
向量u:
\[u =[u_{11}, u_{12}, \dots, u_{1n}, u_{21}, \dots, u_{nn}]\]
记\(s_{ij} = (i^2+j^2)e^{\frac{ij}{100}}\),则向量b:
\[b=[s_{11}, s_{12}, \dots, s_{1n}, s_{21}, \dots, s_{nn}]\]

对于问题(2),我们不妨设\(N = 10\),修改第一题中的迭代程序,计算近似解满足
\(\| u^{(k)} - u^{(k-1)} \Vert < 10^{-5}\)条件的解,结果如下表:
\begin{table}[!ht]
    \begin{tabular*}{\hsize}{@{}@{\extracolsep{\fill}}cc@{}}
    \toprule
    方法 &迭代次数\\
    \midrule
    Jacobi迭代法 & 371\\
    SOR迭代法\(\omega = 1\) & 196 \\
    SOR迭代法\(\omega = 1.25\) & 118\\
    SOR迭代法\(\omega = 1.50\) & 59\\
    SOR迭代法\(\omega = 1.75\) & 69\\
    \bottomrule
    \end{tabular*}
\end{table}
针对问题(2)中要求的条件,我们使用C-G方法,求得迭代次数为32次,由此得到C-G方法的收敛速度远远快于Jacobi方法和SOR方法。




\end{document}